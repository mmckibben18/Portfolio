% Options for packages loaded elsewhere
\PassOptionsToPackage{unicode}{hyperref}
\PassOptionsToPackage{hyphens}{url}
%
\documentclass[
]{article}
\usepackage{amsmath,amssymb}
\usepackage{iftex}
\ifPDFTeX
  \usepackage[T1]{fontenc}
  \usepackage[utf8]{inputenc}
  \usepackage{textcomp} % provide euro and other symbols
\else % if luatex or xetex
  \usepackage{unicode-math} % this also loads fontspec
  \defaultfontfeatures{Scale=MatchLowercase}
  \defaultfontfeatures[\rmfamily]{Ligatures=TeX,Scale=1}
\fi
\usepackage{lmodern}
\ifPDFTeX\else
  % xetex/luatex font selection
\fi
% Use upquote if available, for straight quotes in verbatim environments
\IfFileExists{upquote.sty}{\usepackage{upquote}}{}
\IfFileExists{microtype.sty}{% use microtype if available
  \usepackage[]{microtype}
  \UseMicrotypeSet[protrusion]{basicmath} % disable protrusion for tt fonts
}{}
\makeatletter
\@ifundefined{KOMAClassName}{% if non-KOMA class
  \IfFileExists{parskip.sty}{%
    \usepackage{parskip}
  }{% else
    \setlength{\parindent}{0pt}
    \setlength{\parskip}{6pt plus 2pt minus 1pt}}
}{% if KOMA class
  \KOMAoptions{parskip=half}}
\makeatother
\usepackage{xcolor}
\usepackage[margin=1in]{geometry}
\usepackage{color}
\usepackage{fancyvrb}
\newcommand{\VerbBar}{|}
\newcommand{\VERB}{\Verb[commandchars=\\\{\}]}
\DefineVerbatimEnvironment{Highlighting}{Verbatim}{commandchars=\\\{\}}
% Add ',fontsize=\small' for more characters per line
\usepackage{framed}
\definecolor{shadecolor}{RGB}{248,248,248}
\newenvironment{Shaded}{\begin{snugshade}}{\end{snugshade}}
\newcommand{\AlertTok}[1]{\textcolor[rgb]{0.94,0.16,0.16}{#1}}
\newcommand{\AnnotationTok}[1]{\textcolor[rgb]{0.56,0.35,0.01}{\textbf{\textit{#1}}}}
\newcommand{\AttributeTok}[1]{\textcolor[rgb]{0.13,0.29,0.53}{#1}}
\newcommand{\BaseNTok}[1]{\textcolor[rgb]{0.00,0.00,0.81}{#1}}
\newcommand{\BuiltInTok}[1]{#1}
\newcommand{\CharTok}[1]{\textcolor[rgb]{0.31,0.60,0.02}{#1}}
\newcommand{\CommentTok}[1]{\textcolor[rgb]{0.56,0.35,0.01}{\textit{#1}}}
\newcommand{\CommentVarTok}[1]{\textcolor[rgb]{0.56,0.35,0.01}{\textbf{\textit{#1}}}}
\newcommand{\ConstantTok}[1]{\textcolor[rgb]{0.56,0.35,0.01}{#1}}
\newcommand{\ControlFlowTok}[1]{\textcolor[rgb]{0.13,0.29,0.53}{\textbf{#1}}}
\newcommand{\DataTypeTok}[1]{\textcolor[rgb]{0.13,0.29,0.53}{#1}}
\newcommand{\DecValTok}[1]{\textcolor[rgb]{0.00,0.00,0.81}{#1}}
\newcommand{\DocumentationTok}[1]{\textcolor[rgb]{0.56,0.35,0.01}{\textbf{\textit{#1}}}}
\newcommand{\ErrorTok}[1]{\textcolor[rgb]{0.64,0.00,0.00}{\textbf{#1}}}
\newcommand{\ExtensionTok}[1]{#1}
\newcommand{\FloatTok}[1]{\textcolor[rgb]{0.00,0.00,0.81}{#1}}
\newcommand{\FunctionTok}[1]{\textcolor[rgb]{0.13,0.29,0.53}{\textbf{#1}}}
\newcommand{\ImportTok}[1]{#1}
\newcommand{\InformationTok}[1]{\textcolor[rgb]{0.56,0.35,0.01}{\textbf{\textit{#1}}}}
\newcommand{\KeywordTok}[1]{\textcolor[rgb]{0.13,0.29,0.53}{\textbf{#1}}}
\newcommand{\NormalTok}[1]{#1}
\newcommand{\OperatorTok}[1]{\textcolor[rgb]{0.81,0.36,0.00}{\textbf{#1}}}
\newcommand{\OtherTok}[1]{\textcolor[rgb]{0.56,0.35,0.01}{#1}}
\newcommand{\PreprocessorTok}[1]{\textcolor[rgb]{0.56,0.35,0.01}{\textit{#1}}}
\newcommand{\RegionMarkerTok}[1]{#1}
\newcommand{\SpecialCharTok}[1]{\textcolor[rgb]{0.81,0.36,0.00}{\textbf{#1}}}
\newcommand{\SpecialStringTok}[1]{\textcolor[rgb]{0.31,0.60,0.02}{#1}}
\newcommand{\StringTok}[1]{\textcolor[rgb]{0.31,0.60,0.02}{#1}}
\newcommand{\VariableTok}[1]{\textcolor[rgb]{0.00,0.00,0.00}{#1}}
\newcommand{\VerbatimStringTok}[1]{\textcolor[rgb]{0.31,0.60,0.02}{#1}}
\newcommand{\WarningTok}[1]{\textcolor[rgb]{0.56,0.35,0.01}{\textbf{\textit{#1}}}}
\usepackage{graphicx}
\makeatletter
\def\maxwidth{\ifdim\Gin@nat@width>\linewidth\linewidth\else\Gin@nat@width\fi}
\def\maxheight{\ifdim\Gin@nat@height>\textheight\textheight\else\Gin@nat@height\fi}
\makeatother
% Scale images if necessary, so that they will not overflow the page
% margins by default, and it is still possible to overwrite the defaults
% using explicit options in \includegraphics[width, height, ...]{}
\setkeys{Gin}{width=\maxwidth,height=\maxheight,keepaspectratio}
% Set default figure placement to htbp
\makeatletter
\def\fps@figure{htbp}
\makeatother
\setlength{\emergencystretch}{3em} % prevent overfull lines
\providecommand{\tightlist}{%
  \setlength{\itemsep}{0pt}\setlength{\parskip}{0pt}}
\setcounter{secnumdepth}{-\maxdimen} % remove section numbering
\ifLuaTeX
  \usepackage{selnolig}  % disable illegal ligatures
\fi
\usepackage{bookmark}
\IfFileExists{xurl.sty}{\usepackage{xurl}}{} % add URL line breaks if available
\urlstyle{same}
\hypersetup{
  pdftitle={McKibben\_DSC520\_Ex\_7.2},
  pdfauthor={Makayla McKibben},
  hidelinks,
  pdfcreator={LaTeX via pandoc}}

\title{McKibben\_DSC520\_Ex\_7.2}
\author{Makayla McKibben}
\date{2024-07-28}

\begin{document}
\maketitle

\begin{Shaded}
\begin{Highlighting}[]
\CommentTok{\# Install appropriate packages}
\CommentTok{\# install.packages("tidyverse")}

\CommentTok{\# Importing the data set}
\NormalTok{survey }\OtherTok{\textless{}{-}} \FunctionTok{read.csv}\NormalTok{(}\AttributeTok{file =} \StringTok{\textquotesingle{}student{-}survey.csv\textquotesingle{}}\NormalTok{, }\AttributeTok{header =} \ConstantTok{TRUE}\NormalTok{, }\AttributeTok{sep =}\StringTok{","}\NormalTok{, }\AttributeTok{stringsAsFactors =} \ConstantTok{FALSE}\NormalTok{)}

\CommentTok{\# Looking at a section of the data set to understand the structure}
\FunctionTok{head}\NormalTok{(survey)}
\end{Highlighting}
\end{Shaded}

\begin{verbatim}
##   TimeReading TimeTV Happiness Gender
## 1           1     90     86.20      1
## 2           2     95     88.70      0
## 3           2     85     70.17      0
## 4           2     80     61.31      1
## 5           3     75     89.52      1
## 6           4     70     60.50      1
\end{verbatim}

\begin{Shaded}
\begin{Highlighting}[]
\CommentTok{\# Calling relevant library}
\FunctionTok{library}\NormalTok{(ggplot2)}
\end{Highlighting}
\end{Shaded}

\begin{verbatim}
## Warning: package 'ggplot2' was built under R version 4.4.1
\end{verbatim}

\begin{Shaded}
\begin{Highlighting}[]
\CommentTok{\# Creating plots}
\NormalTok{r\_v\_tv }\OtherTok{\textless{}{-}} \FunctionTok{ggplot}\NormalTok{(survey, }\FunctionTok{aes}\NormalTok{(TimeReading, TimeTV))}
\NormalTok{r\_v\_tv }\SpecialCharTok{+} \FunctionTok{geom\_point}\NormalTok{(}\AttributeTok{color =} \StringTok{"navy"}\NormalTok{, }\AttributeTok{shape =} \DecValTok{8}\NormalTok{, }\AttributeTok{size =} \FloatTok{4.8}\NormalTok{) }\SpecialCharTok{+}
  \FunctionTok{theme}\NormalTok{(}\AttributeTok{panel.grid =} \FunctionTok{element\_line}\NormalTok{(}\AttributeTok{color =} \StringTok{"lightgrey"}\NormalTok{, }\AttributeTok{linewidth =} \FloatTok{0.8}\NormalTok{, }\AttributeTok{linetype =} \DecValTok{1}\NormalTok{), }
        \AttributeTok{panel.background =} \FunctionTok{element\_rect}\NormalTok{(}\AttributeTok{color =} \StringTok{"white"}\NormalTok{, }\AttributeTok{fill =} \StringTok{"darkgrey"}\NormalTok{)) }\SpecialCharTok{+}
  \FunctionTok{labs}\NormalTok{(}\AttributeTok{title =} \StringTok{"Time Spent"}\NormalTok{, }\AttributeTok{x =}\StringTok{"Time Reading"}\NormalTok{, }
       \AttributeTok{y =} \StringTok{"Time TV"}\NormalTok{) }\SpecialCharTok{+} \FunctionTok{xlim}\NormalTok{(}\DecValTok{0}\NormalTok{,}\DecValTok{7}\NormalTok{)}
\end{Highlighting}
\end{Shaded}

\includegraphics{McKibben_Assignment_7.2_files/figure-latex/plots-1.pdf}

\begin{Shaded}
\begin{Highlighting}[]
\NormalTok{happy\_reader }\OtherTok{\textless{}{-}}\FunctionTok{ggplot}\NormalTok{(survey, }\FunctionTok{aes}\NormalTok{(TimeReading, Happiness))}
\NormalTok{happy\_reader }\SpecialCharTok{+} \FunctionTok{geom\_point}\NormalTok{(}\AttributeTok{color =} \StringTok{"purple"}\NormalTok{, }\AttributeTok{shape =} \DecValTok{8}\NormalTok{, }\AttributeTok{size =} \FloatTok{4.8}\NormalTok{) }\SpecialCharTok{+} 
  \FunctionTok{theme}\NormalTok{(}\AttributeTok{panel.grid =} \FunctionTok{element\_line}\NormalTok{(}\AttributeTok{color =} \StringTok{"lightgrey"}\NormalTok{, }\AttributeTok{linewidth =} \FloatTok{0.8}\NormalTok{, }\AttributeTok{linetype =} \DecValTok{1}\NormalTok{),}
        \AttributeTok{panel.background =} \FunctionTok{element\_rect}\NormalTok{(}\AttributeTok{color =} \StringTok{"white"}\NormalTok{, }\AttributeTok{fill =} \StringTok{"darkgrey"}\NormalTok{)) }\SpecialCharTok{+}
  \FunctionTok{labs}\NormalTok{(}\AttributeTok{title =} \StringTok{"Happy Readers"}\NormalTok{, }\AttributeTok{x =}\StringTok{"Time Reading"}\NormalTok{, }
       \AttributeTok{y =} \StringTok{"Happiness"}\NormalTok{) }\SpecialCharTok{+} \FunctionTok{xlim}\NormalTok{(}\DecValTok{0}\NormalTok{,}\DecValTok{7}\NormalTok{) }\SpecialCharTok{+} \FunctionTok{ylim}\NormalTok{(}\DecValTok{45}\NormalTok{,}\DecValTok{95}\NormalTok{)}
\end{Highlighting}
\end{Shaded}

\includegraphics{McKibben_Assignment_7.2_files/figure-latex/plots-2.pdf}

\begin{Shaded}
\begin{Highlighting}[]
\NormalTok{happy\_tv }\OtherTok{\textless{}{-}}\FunctionTok{ggplot}\NormalTok{(survey, }\FunctionTok{aes}\NormalTok{(TimeTV, Happiness))}
\NormalTok{happy\_tv }\SpecialCharTok{+} \FunctionTok{geom\_point}\NormalTok{(}\AttributeTok{color =} \StringTok{"darkgreen"}\NormalTok{, }\AttributeTok{shape =} \DecValTok{8}\NormalTok{, }\AttributeTok{size =} \FloatTok{4.8}\NormalTok{) }\SpecialCharTok{+} 
  \FunctionTok{theme}\NormalTok{(}\AttributeTok{panel.grid =} \FunctionTok{element\_line}\NormalTok{(}\AttributeTok{color =} \StringTok{"lightgrey"}\NormalTok{, }\AttributeTok{linewidth =} \FloatTok{0.8}\NormalTok{, }\AttributeTok{linetype =} \DecValTok{1}\NormalTok{), }
        \AttributeTok{panel.background =} \FunctionTok{element\_rect}\NormalTok{(}\AttributeTok{color =} \StringTok{"white"}\NormalTok{, }\AttributeTok{fill =} \StringTok{"darkgrey"}\NormalTok{)) }\SpecialCharTok{+}
  \FunctionTok{labs}\NormalTok{(}\AttributeTok{title =} \StringTok{"Happy TV"}\NormalTok{, }\AttributeTok{x =}\StringTok{"Time TV"}\NormalTok{, }
       \AttributeTok{y =} \StringTok{"Happiness"}\NormalTok{) }\CommentTok{\#+ xlim(0,7) + ylim(45,95)}
\end{Highlighting}
\end{Shaded}

\includegraphics{McKibben_Assignment_7.2_files/figure-latex/plots-3.pdf}
Exercise 7.2 Q3 TimeReading vs.~TimeTV shows a strong negative
correlation TimeReading vs.~Happiness shows a less strong, less negative
correlation TimeTV vs.~Happiness shows a weak positive correlation

\begin{Shaded}
\begin{Highlighting}[]
\CommentTok{\# Find the covariance matrix}
\NormalTok{data\_group }\OtherTok{\textless{}{-}} \FunctionTok{cbind}\NormalTok{(survey}\SpecialCharTok{$}\NormalTok{TimeReading, survey}\SpecialCharTok{$}\NormalTok{TimeTV, survey}\SpecialCharTok{$}\NormalTok{Happiness)}
\NormalTok{cov\_data }\OtherTok{\textless{}{-}} \FunctionTok{cov}\NormalTok{(data\_group)}
\NormalTok{cov\_data}
\end{Highlighting}
\end{Shaded}

\begin{verbatim}
##            [,1]      [,2]      [,3]
## [1,]   3.054545 -20.36364 -10.35009
## [2,] -20.363636 174.09091 114.37727
## [3,] -10.350091 114.37727 185.45142
\end{verbatim}

Exercise 7.2 Q4 \# TimeReading has a negative covariance with both
TimeTV and Happiness. Happiness decreases half as fast as TimeTV as
TimeReading increases

\section{As TimeTV increases Happiness increases rapidly as they have a
relatively large positive
covariance}\label{as-timetv-increases-happiness-increases-rapidly-as-they-have-a-relatively-large-positive-covariance}

\begin{Shaded}
\begin{Highlighting}[]
\NormalTok{cor\_data }\OtherTok{\textless{}{-}} \FunctionTok{cor}\NormalTok{(data\_group)}
\NormalTok{cor\_data}
\end{Highlighting}
\end{Shaded}

\begin{verbatim}
##            [,1]       [,2]       [,3]
## [1,]  1.0000000 -0.8830677 -0.4348663
## [2,] -0.8830677  1.0000000  0.6365560
## [3,] -0.4348663  0.6365560  1.0000000
\end{verbatim}

Exercise 7.2 Q5 \# TimeReading has a negative correlation with both
TimeTV and Happiness. Happiness decreases half as fast as TimeTV as
TimeReading increases

\section{As TimeTV increases Happiness increases rapidly as they have a
relatively large positive
correlation}\label{as-timetv-increases-happiness-increases-rapidly-as-they-have-a-relatively-large-positive-correlation}

\section{I think that correlation is better and easier to interpret for
one primary reason. I believe it's better because there's a reference
number other than zero i.e.~it's bounded by 1 and
-1}\label{i-think-that-correlation-is-better-and-easier-to-interpret-for-one-primary-reason.-i-believe-its-better-because-theres-a-reference-number-other-than-zero-i.e.-its-bounded-by-1-and--1}

\begin{Shaded}
\begin{Highlighting}[]
\CommentTok{\# Find the corellation between TimeReading and TimeTV}
\NormalTok{cor\_r\_tv }\OtherTok{\textless{}{-}} \FunctionTok{cor}\NormalTok{(survey}\SpecialCharTok{$}\NormalTok{TimeReading, survey}\SpecialCharTok{$}\NormalTok{TimeTV)}
\NormalTok{cor\_r\_tv}
\end{Highlighting}
\end{Shaded}

\begin{verbatim}
## [1] -0.8830677
\end{verbatim}

\section{Exercise 7.2 Q6}\label{exercise-7.2-q6}

\section{TimeReading has a strong negative correlation with
TimeTV.}\label{timereading-has-a-strong-negative-correlation-with-timetv.}

\section{We cannot assume causation from correlation. We cannot assume
causation from correlation. We cannot assume causation from correlation.
I would imagine we could still say that if you spend more time reading
there's less time to spend watching TV, so it probably does have an
effect.}\label{we-cannot-assume-causation-from-correlation.-we-cannot-assume-causation-from-correlation.-we-cannot-assume-causation-from-correlation.-i-would-imagine-we-could-still-say-that-if-you-spend-more-time-reading-theres-less-time-to-spend-watching-tv-so-it-probably-does-have-an-effect.}

\end{document}
